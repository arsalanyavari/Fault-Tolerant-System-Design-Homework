\documentclass[12pt]{article}
 \usepackage{xcolor}
 \usepackage[margin=1in]{geometry} 
 \usepackage{amsmath,amsthm,amssymb}
 
\newcommand{\N}{\mathbb{N}}
\newcommand{\Z}{\mathbb{Z}}
 
\newenvironment{theorem}[2][Theorem]{\begin{trivlist}
\item[\hskip \labelsep {\bfseries #1}\hskip \labelsep {\bfseries #2.}]}{\end{trivlist}}
\newenvironment{lemma}[2][Lemma]{\begin{trivlist}
\item[\hskip \labelsep {\bfseries #1}\hskip \labelsep {\bfseries #2.}]}{\end{trivlist}}
\newenvironment{exercise}[2][Exercise]{\begin{trivlist}
\item[\hskip \labelsep {\bfseries #1}\hskip \labelsep {\bfseries #2.}]}{\end{trivlist}}
\newenvironment{problem}[2][Problem]{\begin{trivlist}
\item[\hskip \labelsep {\bfseries #1}\hskip \labelsep {\bfseries #2.}]}{\end{trivlist}}
\newenvironment{question}[2][Question]{\begin{trivlist}
\item[\hskip \labelsep {\bfseries #1}\hskip \labelsep {\bfseries #2.}]}{\end{trivlist}}
\newenvironment{corollary}[2][Corollary]{\begin{trivlist}
\item[\hskip \labelsep {\bfseries #1}\hskip \labelsep {\bfseries #2.}]}{\end{trivlist}}

\newenvironment{solution}{\begin{proof}[Solution]}{\end{proof}}
 
\begin{document}
 
% --------------------------------------------------------------
%                         Start here
% --------------------------------------------------------------
 
\title{Homework 2}
\author{Amir Arsalan Yavari}
% \date{} 
\maketitle

\textbf{Prove that if the reliability formula is exponential, then the failure of the system is memoryless.}\\[10pt]

\begin{center}
$R(t) = e^{-\lambda t} \implies P(T > t) = P(T > t+s \mid T > s)$\\[7pt]
\end{center}

\textbf{Proof:}

To show that the system is memoryless, we need to prove the following expression:

\begin{center}
    \( P(T > t) = P(T > t+s \mid T > s) \)
\end{center}

First of all we expand the following expression:

\begin{center}
    \( P(T > t+s \mid T > s) = \frac{P(T > t+s \cap T > s)}{P(T > s)} \)
\end{center}

\colorbox{yellow}{We know \( T > t+s \) then \( T > s \). Thus we can say \( P(T > t+s \cap T > s) = P(T > t+s) \).}
\\

Therefore:

\begin{center}
    \( P(T > t+s \mid T > s) = \frac{P(T > t+s)}{P(T > s)} \)
\end{center}

On the one hand, we have the reliability formula, which is the assumption of the problem.

\begin{center}
    \( R(t) = e^{-\lambda t} \implies P(T > t) = e^{-\lambda t} \)
\end{center}

Thus:

\begin{center}
    \( P(T > s) = e^{-\lambda s} \quad \text{and} \quad P(T > t+s) = e^{-\lambda(t+s)} = e^{-\lambda t} \cdot e^{-\lambda s} \)
\end{center}

Now we put the expression \( P(T > t) \) and \( P(T > t+s \cap T > s) = P(T > t+s) \) in reliability formula:

\begin{center}
    \textcolor{teal}{\( P(T > t+s \mid T > s) = \frac{e^{-\lambda(t+s)}}{e^{-\lambda s}} = e^{-\lambda t} \)}
\\[10pt]

    \textcolor{teal}{\( P(T > t) = e^{-\lambda t} \)}
\end{center}

Thus we prove:

\begin{center}
    \colorbox{lime}{\( P(T > t) = P(T > t+s \mid T > s) \)}
\end{center}

\end{document}

